\newglossaryentry{abi}
{
	name={ABI},
	first={\acrlong{abi} (ABI)},
	long={Application Binary Interface},
	description={{\em \acrlong{acm}} A computer programming
interface provided by libraries, for example.  This interface promises to
the developers a certain behaviour, including the alignment of data types
and calling conventions.  See also: API}
}

\newglossaryentry{acl}
{
	name={ACL},
	first={\acrlong{acl} (ACL)},
	long={Access Control List},
	description={{\em \acrlong{acl}}  A list of
permissions, commonly attached to a network or file resources.
An \acrshort{acm} defines e.g. what privileges are granted to
the object or what sort of network traffic may be
allowed.}
}

\newglossaryentry{acm}
{
	name={ACM},
	first={\acrlong{acm} (ACM)},
	long={Association for Computing Machinery},
	description={{\em \acrlong{acm}}  A learned society for
computing, governing Special Interest Groups, publishing academic journals
and sponsoring numerous conferences.}
}

\newglossaryentry{afp}
{
	name={AFP},
	first={\acrlong{afp} (AFP)},
	long={Apple Filing Protocol},
	description={{\em \acrlong{afp}}   A network file system / protocol
used predominantly by Apple's Mac OS versions (both ``Classic'' and OS X).
Sometimes also referred to as ``Apple Share''.}
}

\newglossaryentry{afs}
{
	name={AFS},
	first={\acrlong{afs} (AFS)},
	long={Andrew File System},
	description={{\em \acrlong{afs}} A distributed network file system
developed at Carnegie Mellon University with certain distinguishing
features over other network file systems, including authentication via
Kerberos, access control lists and, to some degree,
location independence.}
}

\newglossaryentry{api}
{
	name={API},
	first={\acrlong{api} (API)},
	description={{\em \acrlong{api}} A specification
describing the interfaces of a given software component.  This
specification is sufficiently high-level so as to allow the developer of
software using this API to not care about {\em how},
the interface is implemented. See also: ABI},
	long={Application Programming Interface}
}

\newglossaryentry{aws}
{
	name={AWS},
	first={\acrlong{aws} (AWS)},
	long={Amazon Web Services},
	description={{\em \acrlong{aws}} Amazon.com's cloud computing
platform.}
}

\newglossaryentry{bbs}
{
	name={BBS},
	first={\acrlong{bbs} (BBS)},
	long={Bulletin Board System},
	description={{\em \acrlong{bbs}} The electronic version of a
traditional bulleting board and the predecessor of Usenet and,
conceptually, todays Internet ``forums''.}
}

\newglossaryentry{bind}
{
	name={BIND},
	first={\acrlong{bind} (BIND)},
	long={Berkeley Internet Name Domain},
	description={{\em \acrlong{bind}} The most widely used DNS
software, included in various Unix flavors since
4.3BSD.}
}

\newglossaryentry{bios}
{
	name={BIOS},
	first={\acrlong{bios} (BIOS)},
	long={Basic Input/Output System},
	description={{\em \acrlong{bios}} The basic firmware found on
IBM compatible computers, loaded from read-only memory at system startup
and in charge of initializing some of the hardware components before
handing control over to the boot loader.}
}

\newglossaryentry{bsd}
{
	name={BSD},
	first={\acrlong{bsd} (BSD)},
	long={Berkeley Software Distribution},
	description={{\em \acrlong{bsd}} Commonly refers to a number of
UNIX derivatives based on software distributed by the Computer Systems
Research Group of the University of California at Berkeley.  Open Source
versions include NetBSD, FreeBSD and OpenBSD.}
}

\newglossaryentry{chs}
{
	name={CHS},
	first={\acrlong{chs} (CHS)},
	long={Cylinder-Head-Sector},
	description={{\em \acrlong{chs}} An early scheme used to address
individual sectors (or blocks) on a hard drive by identifying their
location on the disk via the given cylinder (or track), the disk's
read-write heads and finally the individual sector.}
}

\newglossaryentry{cifs}
{
	name={CIFS},
	first={\acrlong{cifs} (CIFS)},
	long={Common Internet File System},
	description={{\em \acrlong{cifs}} An application layer network
protocol used predominantly by Microsoft Windows systems as a network file
system and to communciate with other devices, such as shared printers.
Frequently named {\em SMB/CIFS}, as it was derived from SMB.}
}

\newglossaryentry{cli}
{
	name={CLI},
	first={\acrlong{cli} (CLI)},
	long={Command Line Interface},
	description={{\em \acrlong{cli}} A
human-computer interface primarily relying on
text-based input whereby the user types commands into
a terminal or console interface.  The CLI is valued by
System Administrators as being widely more efficient
and flexible than most graphical interfaces.  See
also: GUI}
}


\newglossaryentry{cm}
{
	name={CM},
	first={\acrlong{cm} (CM)},
	long={Configuration Management},
	description={{\em \acrlong{cm}} See SCM}
}

\newglossaryentry{cpu}
{
	name={CPU},
	first={\acrlong{cpu} (CPU)},
	long={Central Processing Unit},
	description={{\em \acrlong{cpu}} The actual
circuitry (e.g. a microprocessor chip) of a computer
performing the arithmetic and logical operations
within a computer, possibly comprising multiple
computing components or ``cores'' and nowadays even
extending to virtual CPUs.}
}
\newglossaryentry{csnet}
{
	name={CSNET},
	first={\acrlong{csnet} (CSNET)},
	long={Computer Science Network},
	description={{\em \acrlong{csnet}} An early network connecting
computer science departments at academic and research
institutions.}
}

\newglossaryentry{csrg}
{
	name={CSRG},
	first={\acrlong{csrg} (CSRG)},
	long={Computer Systems Research Group},
	description={{\em \acrlong{csrg}} A research group at the
University of California, Berkeley, funded by the Defense Advanced
Research Projects Agency to enhance the Unix operating
system.}
}

\newglossaryentry{cvs}
{
	name={CVS},
	first={\acrlong{cvs} (CVS)},
	long={Concurrent Versions System},
	description={{\em \acrlong{cvs}} The at one point perhaps most
popular open source client-server version control system.  (Yes, {\em CVS}
is a {\em VCS}.  One thing we cannot complain about is a lack of
acronyms.)  In recent years, CVS has lost popularity to the Subversion VCS
and newer distributed revision control systems such as
Git.}
}

\newglossaryentry{darpa}
{
	name={DARPA},
	first={\acrlong{darpa} (DARPA)},
	long={Defense Advanced Research Projects Agency},
	description={{\em \acrlong{darpa}} Originally
known as ``ARPA'', this agency of the United States Department of Defense
is probably best known for the development of what ultimately became the
Internet.}
}

\newglossaryentry{das}
{
	name={DAS},
	first={\acrlong{das} (DAS)},
	long={Direct Attached Storage},
	description={{\em \acrlong{das}} A storage system in which the
disk devices are attached directly, i.e. without any network component, to
the server.  See also: NAS, SAN}
}

\newglossaryentry{dsl}
{
	name={DSL},
	first={\acrlong{dsl} (DSL)},
	long={Domain Specific Language},
	description={{\em \acrlong{dsl}} A language developed for a very
specific purpose, such as the configuration of a given piece of software
or the representation of information within a certain
data model.}
}

\newglossaryentry{dns}
{
	name={DNS},
	first={\acrlong{dns} (DNS)},
	long={Domain Name System},
	description={{\em \acrlong{dns}} A hierarchical, distributed system to
map hostnames to IP addresses (amongst other things).}
}

\newglossaryentry{dram}
{
	name={DRAM},
	first={\acrlong{dram} (DRAM)},
	long={Dynamic Random Access Memory},
	description={{\em \acrlong{dram}} A volatile type of computer
storage used for most computers' and laptops' main
memory.}
}

\newglossaryentry{ebs}
{
	name={EBS},
	first={\acrlong{ebs} (EBS)},
	long={Elastic Block Store},
	description={{\em \acrlong{ebs}} Amazon's block-level cloud storage service.  See also: S3}
}

\newglossaryentry{ec2}
{
	name={EC2},
	first={\acrlong{ec2} (EC2)},
	long={Elastic Compute Cloud},
	description={{\em \acrlong{ec2}} Part of Amazon's Web Services, EC2
allows a user to deploy virtual machines or ``compute instances'' on
demand.}
}

\newglossaryentry{fcp}
{
	name={FCP},
	first={\acrlong{fcp} (FCP)},
	long={Fibre Channel Protocol},
	description={{\em \acrlong{fcp}} A high-speed network protocol
primarily used to connect components in a Storage Area Networks.  FCP
allows for a number of different topologies, most notably connections in a
switched fabric.}
}

\newglossaryentry{ffs}
{
	name={FFS},
	first={\acrlong{ffs} (FFS)},
	long={Fast File System},
	description={{\em \acrlong{ffs}} See: UFS}
}

\newglossaryentry{gcc}
{
	name={GCC},
	first={\acrlong{gcc} (GCC)},
	long={GNU Compiler Collection},
	description={{\em \acrlong{gcc}} A suite of
tools comprising a compiler {\em chain} provided by
the GNU project.  Originally referring to the GNU C
Compiler, the tools provided include support for many
different languages, including C++, Objective-C,
Fortran, Java, Ada, and Go.  The C compiler, invoked via the
{\tt gcc(1)} command, is the de-facto standard and
is shipped with most Unix flavors. See also: GNU}
}

\newglossaryentry{gnu}
{
	name={GNU},
	first={\acrlong{gnu} (GNU)},
	long={GNU's Not Unix},
	description={{\em \acrlong{gnu}} The GNU project was founded to provide free
software and aimed to provide a full operating system.  After having
adopted the Linux kernel, GNU/Linux become commonly referred to just as
``Linux'', much to the chagrin of many GNU proponents.  The contributions
of the GNU project, however, should not be
underestimated.  See also: GPL}
}

\newglossaryentry{gpl}
{
	name={GPL},
	first={\acrlong{gpl} (GPL)},
	long={GNU General Public License},
	description={{\em \acrlong{gpl}} A widely used free software license
originally written by Richard Stallman for the GNU Project.  The license
aims to guarantee availability of the source code for the licensed
software as well as any derived works.}
}

\newglossaryentry{gui}
{
	name={GUI},
	first={\acrlong{gui} (GUI)},
	long={Graphical User Interface},
	description={{\em \acrlong{gui}} A
human-computer interface primarily relying on
interactions with the user through e.g. a mouse or
other pointing device, icons, and other visual cues.
See also / contrast with: CLI}
}

\newglossaryentry{hba}
{
	name={HBA},
	first={\acrlong{hba} (HBA)},
	long={Host Bus Adapter},
	description={{\em \acrlong{hba}} A hardware connector, such as a PCI,
PCI-X, or PCIe card, connecting, for example, a storage medium to a host
system.}
}

\newglossaryentry{hdd}
{
	name={HDD},
	first={\acrlong{hdd} (HDD)},
	long={Hard Disk Drive},
	description={{\em \acrlong{hdd}} A data
storage device using magnetic storage, frequently
rotating platters allowing for random access via a
magnetic read-write head.  See also: SDD}
}
\newglossaryentry{http}
{
	name={HTTP},
	first={\acrlong{http} (HTTP)},
	long={Hyper Text Transfer Protocol},
	description={{\em \acrlong{http}} The ubiquitous
application-layer protocol underlying the World Wide Web, allowing for
distributed documents to be linked.}
}

\newglossaryentry{iaas}
{
	name={IaaS},
	first={\acrlong{iaas} (IaaS)},
	long={Infrastructure as a Service},
	description={{\em \acrlong{iaas}} A concept in cloud computing
whereby infrastructure components are deployed and delivered on demand,
frequently by use of virtualization.}
}

\newglossaryentry{iana}
{
	name={IANA},
	first={\acrlong{iana} (IANA)},
	long={Internet Assigned Numbers Authority},
	description={{\em \acrlong{iana}} An organization
responsible for the global coordination of the DNS Root (i.e., maintenance of
the DNS root zones), IP addressing (i.e., overseeing global IP address
allocation to the regional Internet registries), and other Internet
Protocol resources.  See also: ICANN}
}

\newglossaryentry{icann}
{
	name={ICANN},
	first={\acrlong{icann} (ICANN)},
	long={Internet Corporation for Assigned Names and Numbers},
	description={{\em \acrlong{icann}} A
non-profit corporation overseeing Internet-related tasks, such as the
operation of the IANA.}
}

\newglossaryentry{ieee}
{
	name={IEEE},
	first={\acrlong{ieee} (IEEE)},
	long={Institute of Electrical and Electronics Engineers},
	description={{\em \acrlong{ieee}} A
non-profit professional organization for the advancement of technological
innovation.  Publishes countless journals, leads the creation of many
standards etc.  Some overlap between the IEEE Computer Society and the
ACM.}
}

\newglossaryentry{ietf}
{
	name={IETF},
	first={\acrlong{ietf} (IETF)},
	long={Internet Engineering Task Force},
	description={{\em \acrlong{ietf}} An open, volunteer-based
organization responsible for the development and creation of Internet
standards.  See also: ISOC}
}

\newglossaryentry{iot}
{
	name={IoT},
	first={\acrlong{iot} (IoT)},
	long={Internet of Things},
	description={{\em \acrlong{iot}}
A term describing the internetworking of devices,
especially consumer products, previously typically not
expected to be connected to the internet.  Such
devices are frequently referred to as ``smart''
devices, despite their tendency to be poorly secured
and with questionable functionality deriving from
their ability to connect to -- and be reached from --
the public internet.}
}

\newglossaryentry{ip}
{
	name={IP},
	first={\acrlong{ip} (IP)},
	long={Internet Protocol},
	description={{\em \acrlong{ip}} The fundamental protocol underlying the
Internet, routing datagrams across networks.  Together
with the \acrlong{ip}, the suite is commonly referred
to as {\em TCP/IP}. See also: TCP, UDP}
}

\newglossaryentry{ipc}
{
	name={IPC},
	first={\acrlong{ipc} (IPC)},
	long={Interprocess Communication},
	description={{\em \acrlong{ipc}} Methods for exchanging data
between related or unrelated processes on one or more
systems.}
}

\newglossaryentry{isoc}
{
	name={ISOC},
	first={\acrlong{isoc} (ISOC)},
	long={Internet Society},
	description={{\em \acrlong{isoc}} An international non-profit
organization providing guidance and direction to Internet related
standards and policy.  Parent organization of the IETF, but contains
additionally a strong focus on education.}
}

\newglossaryentry{iscsi}
{
	name={iSCSI},
	first={\acrlong{iscsi} (iSCSI)},
	long={Internet Small Computer System Interface},
	description={{\em \acrlong{iscsi}} A networking
standard for carrying SCSI commands over IP networks.
See also: IP, SCSI}
}

\newglossaryentry{itil}
{
	name={ITIL},
	first={\acrlong{itil} (ITIL)},
	long={Information Technology Infrastructure Library},
	description={{\em \acrlong{itil}} A set of
practices underlying the first international standard for IT Service
Management, ISO/IEC 20000.  ITIL covers a series of publications focusing
on Service Strategy, Service Design, Service Transition, Service
Operation, and Continual Service Improvement.}
}

\newglossaryentry{jbod}
{
	name={JBOD},
	first={\acrlong{jbod} (JBOD)},
	long={Just a Bunch Of Disks},
	description={{\em \acrlong{jbod}} A term describing a simple storage
configuration where individual disks are made available in the operating
system as separate logical units accessed via separate
mount points.}
}

\newglossaryentry{json}
{
	name={JSON},
	first={\acrlong{json} (JSON)},
	first={\acrlong{json} (JSON)},
	long={JavaScript Object Notation},
	description={{\em \acrlong{json}} An
open standard to describe data
objects using key-value pairs.  Despite its name, the
data format is language independent, and libraries to
process and generate JSON exist for virtually all
common programming languages.
	}
}

\newglossaryentry{lan}
{
	name={lAN},
	first={\acrlong{lan} (LAN)},
	long={Local Area Network},
	description={{\em \acrlong{lan}} A physically
close computer network connecting individual
components on Layer 1 or 2 of the OSI stack. See also:
WAN}
}

\newglossaryentry{lba}
{
	name={LBA},
	first={\acrlong{lba} (LBA)},
	long={Logical Block Addressing},
	description={{\em \acrlong{lba}} A scheme used to address
individual sectors (or blocks) on a hard drive by
iterating over them.}
}

\newglossaryentry{ldap}
{
	name={LDAP},
	first={\acrlong{ldap} (LDAP)},
	long={Lightweight Directory Access Protocol},
	description={{\em \acrlong{ldap}} A common protocol
for accessing directory services, such as username lookups, user
groupings, password storage, and other such data.
Defined in RFC4511.}
}

\newglossaryentry{lisa}
{
	name={LISA},
	first={\acrlong{lisa} (LISA)},
	long={The USENIX Special Interest Group for Sysadmins},
	description={{\em \acrlong{lisa}} A non-profit
organization established to serve the System Administration community.
See also: LOPSA.}
}

\newglossaryentry{lopsa}
{
	name={LOPSA},
	first={\acrlong{lopsa} (LOPSA)},
	long={League of Professional System Administrators},
	description={{\em \acrlong{lopsa}} A non-profit
organization established to advance the profession and practice of System
Administration.  See also: LISA.}
}

\newglossaryentry{lun}
{
	name={LUN},
	first={\acrlong{lun} (LUN)},
	long={Logical Unit Number},
	description={{\em \acrlong{lun}} A numerical identifier for a distinct
storage unit or volume in a Storage Area Network.}
}

\newglossaryentry{lvm}
{
	name={LVM},
	first={\acrlong{lvm} (LVM)},
	long={Logical Volume Manager},
	description={{\em \acrlong{lvm}} A tool or
technique to allocate and manage storage space across
multiple block-storage devices and to present them to
the OS as a virtual block device, possibly increasing
storage capacity or performance. }
}

\newglossaryentry{mbr}
{
	name={MBR},
	first={\acrlong{mbr} (MBR)},
	long={Master Boot Record},
	description={{\em \acrlong{mbr}} A special
boot sector found on the primary boot device, allowing
the system to access the file system and transfer
control to a second-stage bootloader.}
}


\newglossaryentry{multics}
{
	name={Multics},
	first={\acrlong{multics} (Multics)},
	long={Multiplexed Information and Computing Service},
	description={{\em \acrlong{multics}} A time-sharing operating system
initially developed in the 1960s in collaboration amongst MIT, General Electric,
and Bell Labs. See also: Unics}
}

\newglossaryentry{milnet}
{
	name={MILNET},
	first={\acrlong{milnet} (MILNET)},
	long={Military Network},
	description={{\em \acrlong{milnet}} The part of the ARPANET designated
for unclassified communications of the US Department
of Defense.}
}

\newglossaryentry{nas}
{
	name={NAS},
	first={\acrlong{nas} (NAS)},
	long={Network Attached Storage},
	description={{\em \acrlong{nas}} A storage model in which disk
devices are made available over the network by a file server to remote
clients.  The file server is running an operating system and maintains the
file system on the storage media; client access the data over the network
using specific network file system protocols.  See
also: DAS, SAN}
}

\newglossaryentry{nfs}
{
	name={NFS},
	first={\acrlong{nfs} (NFS)},
	long={Network File System},
	description={{\em \acrlong{nfs}} A distributed network file system
developed by Sun Microsystems.  NFS has become the de-facto standard in
distributed file systems in the Unix world and is supported by virtually
all NAS solutions.}
}

\newglossaryentry{nnt}
{
	name={NNT},
	first={\acrlong{nnt} (NNT)},
	long={Network News Transfer Protocol},
	description={{\em \acrlong{nnt}} An application-level
protocol used to transport messages or ``news articles'' for Usenet,
conceptually similar to SMTP.}
}

\newglossaryentry{nsfnet}
{
	name={NSFNET},
	first={\acrlong{nsfnet} (NSFNET)},
	long={National Science Foundation Network},
	description={{\em \acrlong{nsfnet}} A network
supporting the initiatives of the National Science Foundation and
initially connecting a small number of supercomputing, later on developed
into a major part of the Internet backbone.}
}

\newglossaryentry{nsi}
{
	name={NSI},
	first={\acrlong{nsi} (NSI)},
	long={NASA Science Internet},
	description={{\em \acrlong{nsi}} A multiprotocol wide area network,
combining a DECnet and a TCP/IP based network (the Space Physics Analysis
Network or SPAN and the NASA Science Network or NSN,
respectively).}
}

\newglossaryentry{os}
{
	name={OS},
	first={\acrlong{os} (OS)},
	long={Operating System},
	description={{\em \acrlong{os}} \empty}
}

\newglossaryentry{osi}
{
	name={OSI},
	first={\acrlong{osi} (OSI)},
	long={Open Systems Interconnection model},
	description={{\em \acrlong{osi}} A conceptual
model characterizing and expanding on the layers
described in the TCP/IP model.  The OSI model
describes seven layers: Physical, Data, Network,
Transport, Session, Presentation, Application.}
}

\newglossaryentry{pata}
{
	name={PATA},
	first={\acrlong{pata} (PATA)},
	long={Parallel Advanced Technology Attachment},
	description={{\em \acrlong{pata}} An interface
standard for connecting storage devices to a host system.  Initially named
{\em ATA}, it was renamed {\em PATA} to avoid confusion with {\em SATA}.
Also frequently referred to as {\em IDE}.}
}

\newglossaryentry{pci}
{
	name={PCI},
	first={\acrlong{pci} (PCI)},
	long={Peripheral Component Interconnect},
	description={{\em \acrlong{pci}} A computer expansion bus
used to attach a hardware device such as an HBA to a computer.  There are
different standards (PCI, PCI Express, PCI-X) providing different speeds
and features. Not to be confused with {\em PCI DSS}.}
}

\newglossaryentry{pcidss}
{
	name={PCI DSS},
	first={\acrlong{pcidss} (PCIDSS)},
	long={Payment Card Industry Data Security Standard},
	description={{\em \acrlong{pcidss}} The
information security standard describing the requirements a merchant needs
to meet in order to accept credit cards.}
}

\newglossaryentry{pola}
{
	name={POLA},
	first={\acrlong{pola} (POLA)},
	long={Principle of Least Astonishment},
	description={{\em \acrlong{pola}} A concept of
precitability in software tools, following which any invocation should not
lead to any surprises by the user.}
}

\newglossaryentry{posix}
{
	name={POSIX},
	first={\acrlong{posix} (POSIX)},
	long={Portable Operating System Interface},
	description={{\em \acrlong{posix}} A family of IEEE
standards defining the common API and (command-line) interfaces, primarily
used by the Unix family of operating systems.}
}

\newglossaryentry{post}
{
	name={POST},
	first={\acrlong{post} (POST)},
	long={Power-On Self Test},
	description={{\em \acrlong{post}} A number of simple routines intended
to ensure that the hardware is not obviously faulty, run by most server
systems immediately after the system is powered on and before the boot
loader is run.}
}

\newglossaryentry{pxe}
{
	name={PXE},
	first={\acrlong{pxe} (PXE)},
	long={Preboot eXecution Environment},
	description={{\em \acrlong{pxe}} A combination of protocols
that allow a computer to determine its network information and boot media
dynamically so as to allow for bootstrapping the system over the network.
This process is also known as {\em pxebooting}.}
}

\newglossaryentry{radius}
{
	name={RADIUS},
	first={\acrlong{radius} (RADIUS)},
	long={Remote Authentication Dial In User Service},
	description={{\em \acrlong{radius}} A network
protocol allowing clients to be managed with regards to authentication,
authorization, and accounting via a central service.}
}

\newglossaryentry{raid}
{
	name={RAID},
	first={\acrlong{raid} (RAID)},
	long={Redundant Array of Independent Disks},
	description={{\em \acrlong{raid}} A storage
technology that allows multiple disks to be combined into a single data
container upon which a file system can be created.  Different schemas allow
for increased data redundancy or I/O performance at the cost of decreased
capacity.}
}

\newglossaryentry{rest}
{
	name={REST},
	first={\acrlong{rest} (REST)},
	long={REpresentationsl State Transfer},
	description={{\em \acrlong{rest}} A software architecture
granting distributed access to an object model.  As the name suggests, the
focus is on relaying a given object's current {\em state} See also: SOAP}
}

\newglossaryentry{rfc}
{
	name={RFC},
	first={\acrlong{rfc} (RFC)},
	long={Request For Comments},
	description={{\em \acrlong{rfc}} Publications outlining Internet
related technologies, research, protocols etc.  Some of the RFCs may
become actual {\em standards}; many of them are de-facto standards.}
}

\newglossaryentry{sdd}
{
	name={SDD},
	first={\acrlong{sdd} (SDD)},
	long={Solid State Drive},
	description={{\em \acrlong{sdd}} A
storage device using non-volatile Flash memory to
store data.  See also: HDD}
}
\newglossaryentry{s3}
{
	name={S3},
	first={\acrlong{s3} (S3)},
	long={Simple Storage Service},
	description={{\em \acrlong{s3}} Amazon's object-level cloud storage
service.  See also: EBS}
}

\newglossaryentry{san}
{
	name={SAN},
	first={\acrlong{san} (SAN)},
	long={Storage Area Network},
	description={{\em \acrlong{san}} A network providing access to disk
devices on the block level.  The storage is made accessible to remote
clients on a block level; clients can then create a file system on top of
these storage blocks.  See also: DAS, NAS}
}

\newglossaryentry{sata}
{
	name={SATA},
	first={\acrlong{sata} (SATA)},
	long={Serial Advanced Technology Attachment},
	description={{\em \acrlong{sata}} An interface
standard for connecting storage devices to a host system using high-speed
serial cables.  See also: PATA}
}

\newglossaryentry{scsi}
{
	name={SCSI},
	first={\acrlong{scsi} (SCSI)},
	long={Small Computer System Interface},
	description={{\em \acrlong{scsi}} A set of standards for
physically connecting and transferring data between computers and
peripheral devices.  Interfaces for using SCSI are numbersome and range
from so-called ``Fast SCSI'' (parallel) to Fibre
Channel.  See also: iSCSI}
}

\newglossaryentry{scm-cm}
{
	name={SCM},
	first={\acrlong{scm-cm} (SCM)},
	long={Software Configuration Management},
	description={{\em \acrlong{scm}} The task of maintaining
changes to a computer system's runtime configuration without manual
intervention.  SCM systems are able to install software, update existing
or create new files, or run commands across large numbers of servers.
Some systems include support for non-traditional host systems, including
networking equipment.  Examples of popular SCM systems are: CFengine,
Chef, Puppet.}
}

\newglossaryentry{scm}
{
	name={SCM},
	first={\acrlong{scm} (SCM)},
	long={Source Control Management},
	description={{\em \acrlong{scm}} The task of tracking changes
during software development.  Often referred to as {\em revision control},
performed by a {\em Version Control System}, (VCS).  Examples include CVS,
Subversion, Perforce and Git. To avoid confusion with Software
Configuration Management, we will use the acronym VCS when referring to
source control management.}
}

\newglossaryentry{smb}
{
	name={SMB},
	first={\acrlong{smb} (SMB)},
	long={Server Message Block},
	description={{\em \acrlong{smb}} See CIFS.}
}

\newglossaryentry{sla}
{
	name={SLA},
	first={\acrlong{sla} (SLA)},
	long={Service Level Agreement},
	description={{\em \acrlong{sla}} An agreement between service
consumers and providers, outlining the expecations the users of the
service may pose and that the provider is obligated to meet.  Examples
include maximum turnaround time until an issue is resolved, maximum
downtime of a service or minimum throughput or
bandwidth etc.}
}

\newglossaryentry{smtp}
{
	name={SMTP},
	first={\acrlong{smtp} (SMTP)},
	long={Simple Mail Transfer Protocol},
	description={{\em \acrlong{smtp}} An application-level
protocol used to exchange electronic messages, or email, amongst mail
servers.}
}

\newglossaryentry{snmp}
{
	name={SNMP},
	first={\acrlong{snmp} (SNMP)},
	long={Simple Network Monitoring Protocol},
	description={{\em \acrlong{snmp}} The industry standard
protocol used for monitoring network devices, computer, printers and all
sorts of other systems.}
}

\newglossaryentry{soap}
{
	name={SOAP},
	first={\acrlong{soap} (SOAP)},
	long={Simple Object Access Protocol},
	description={{\em \acrlong{soap}} A protocol used by many
web services to exchange structured information over HTTP using XML.  See
also: REST}
}

\newglossaryentry{spof}
{
	name={SPOF},
	first={\acrlong{spof} (SPOF)},
	long={Single Point of Failure},
	description={{\em \acrlong{spof}} A common term describing a
crucial system component without which nothing works.  Note that
{\em people} can easily become a Single Point of Failure if they retain
exclusive knowledge about the system infrastructure or the internal
details of a given program.}
}

\newglossaryentry{sre}
{
	name={SRE},
	first={\acrlong{sre} (SRE)},
	long={Site Reliability Engineering},
	description={{\em \acrlong{sre}} A term
describing the merging of software engineering with
traditional ``operations'' or system administration,
in many ways similar to the term ``DevOps''.  The term
is believed to originate at Google.
	}
}

\newglossaryentry{ssl}
{
	name={SSL},
	first={\acrlong{ssl} (SSL)},
	long={Secure Sockets Layer},
	description={{\em \acrlong{ssl}} The predecessor to TLS, initially
developed by Netscape.  To this day, TLS and SSL are used interchangeably,
even though TLS has superseded SSL.}
}

\newglossaryentry{tacacs+}
{
	name={TACACS+},
	first={\acrlong{tacacs} (TACACS)},
	long={Terminal Access Controller Access-Control System Plus},
	description={{\em \acrlong{tacacs+}} A remote authentication protocol often used for network equipment,
allowing routers and switches, for example, to communicate with a central
authentication server.}
}

\newglossaryentry{tco}
{
	name={TCO},
	first={\acrlong{tco} (TCO)},
	long={Total Cost of Ownership},
	description={{\em \acrlong{tco}} In software engineering, the total
cost of ownership provides an estimate of what it takes to build or run a
system.  This includes the initial development or purchase cost as well as
the ongoing cost (both monetary as well as in human resources) to
maintain, update, patch and debug the software.}
}

\newglossaryentry{tcp}
{
	name={TCP},
	first={\acrlong{tcp} (TCP)},
	long={Transmission Control Protocol},
	description={{\em \acrlong{tcp}} One of the fundamental protocols underlying the
Internet, routing datagrams across networks.  Together
with the \acrlong{ip}, the suite is commonly referred
to as {\em TCP/IP}.  TCP provides for reliable,
connection-oriented connections between e.g. a server
and a client.  See also: IP, UDP}
}

\newglossaryentry{tls}
{
	name={TLS},
	first={\acrlong{tls} (TLS)},
	long={Transport Layer Security},
	description={{\em \acrlong{tls}} A protocol to allow for
encryption of network traffic on the application layer, currently defined
in RFC5246.  TLS was initially based on SSL, with which it is frequently
used interchangeably.}
}

\newglossaryentry{tofu}
{
	name={TOFU},
	first={\acrlong{tofu} (TOFU)},
	long={Trust on First Use},
	description={{\em \acrlong{tofu}} In secure
communications, it may be difficult or impossible to
verify the identify of a remote party prior to e.g.
establishing a connection.  After an initial
communication, a trust identifier may be exchanged,
which the client might store locally and then report
any possible mismatches on subsequent connections.
This act of initially accepting the token without
explicit verification is then referred to as {\em
Trust on First Use}.  A common example might be a user
accepting the ssh hostkey fingerprint of a host they
have not previously connected to.
}
}

\newglossaryentry{udp}
{
	name={UDP},
	first={\acrlong{udp} (UDP)},
	long={User Datagram Protocol},
	description={{\em \acrlong{udp}} A transport
layer protocol (i.e. operating on layer 3 of the
TCP/IP model, layer 4 of the OSI model), which allows
for connectionless, unreliable communications between
two endpoints.  See also: IP, TCP}
}

\newglossaryentry{uefi}
{
	name={UEFI},
	first={\acrlong{uefi} (UEFI)},
	long={Unified Extensible Firmware Interface},
	description={{\em \acrlong{uefi}} A specification
defining the interactions between an Operating System and lower-level
firmware.  See also: BIOS}
}

\newglossaryentry{ufs}
{
	name={UFS},
	first={\acrlong{ufs} (UFS)},
	long={Unix File System},
	description={{\em \acrlong{ufs}} A widely adopted file system across
different Unix versions, implementing boot blocks, superblocks, cylinder
groups, inodes and data blocks.  Also called the Berkeley Fast File System
(FFS).}
}

\newglossaryentry{unics}
{
	name={Unics},
	first={\acrlong{unics} (Unics)},
	long={Uniplexed Information and Computing Service},
	description={{\em \acrlong{unics}} The
original name of the UNIX operating system, a pun on
``Multics''.}
}

\newglossaryentry{vcs}
{
	name={VCS},
	first={\acrlong{vcs} (VCS)},
	long={Version Control System},
	description={{\em \acrlong{vcs}} A
component of software and configuration management,
allowing for the tracking of revisions made to a given
set of files.  See also: CVS, SCM}
}

\newglossaryentry{vfs}
{
	name={VFS},
	first={\acrlong{vfs} (VFS)},
	long={Virtual File System},
	description={{\em \acrlong{vfs}} An abstraction layer providing
applications with a consistent interface to different underlying
file systems.  Initially developed by Sun Microsystems.  The term {\tt
vnode} indicates this heritage.}
}

\newglossaryentry{vps}
{
	name={VPS},
	first={\acrlong{vps} (VPS)},
	long={Virtual Private Server},
	description={{\em \acrlong{vps}} A virtual machine provided by a
hosting company to customers.  Payment models range from hourly to monthly
or longer subscriptions.}
}

\newglossaryentry{wan}
{
	name={WAN},
	first={\acrlong{wan} (WAN)},
	long={Wide Area Network},
	description={{\em \acrlong{wan}} A computer
network spanning wide-ranging regions, possibly
connecting different \acrlong{lan}s, designed to
overcome physical proximity requirements. See also: LAN}
}

\newglossaryentry{xml}
{
	name={XML},
	first={\acrlong{xml} (XML)},
	long={eXtensible Markup Language},
	description={{\em \acrlong{xml}} A markup
language based on free, open standards that is frequently
used in web services or by system tools to
describe complex configuration attributes of a
service.}
}
